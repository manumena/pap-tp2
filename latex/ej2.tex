\section{Ejercicio 2}

\subsection{Introducción}

El objetivo de este problema era dado un número $A$ de acciones con sus valores
en el transcurdo de $D$ días devolver el mínimo número de gráficos necesarios
para graficarlas sin que hubiera intersecciones entre las visualizaciones de las
mismas.

\subsubsection{Solución propuesta}

La resolución de este ejercicio cuenta con varias etapas. La primera consistió
en modelar la entrada de forma tal que fuera posible visualizar en otros
términos el problema a resolver.

Para esto se decidió representar las acciones mediante un grafo acíclico
dirigido. El modo de construirlo fue el siguiente:

\begin{itemize}
	\item{Cada acción es representada mediante un vértice.}
	\item{Un arco saliendo de una acción $A_j$ a $A_k$ implica que el valor de
	la acción $A_j$ se encuentra acotado superiormente por el de $A_k$ en todo
	momento.}
\end{itemize}

Se puede ver fácilmente que no habrá ciclos puesto que ello implicaría que el
valor de una acción acota superior e inferiormente el de otra, lo cual sería
un absurdo.

A modo de ejemple se presenta a continuación una posible entrada con su
respectiva representación en el modelo propuesto.

\newcolumntype{M}[1]{>{\centering\arraybackslash}m{#1}}

\begin{table}[H]
	\caption{Ejemplo de entrada con $A = 5$ y $D = 3$.} \label{ej2:tab1}
	\centering
	\begin{tabular}{|c|M{2em}|M{2em}|M{2em}|}
		\hline
		\multirow{2}{4em}{\centering Acción} & \multicolumn{3}{c|}{Valor en día $i$} \\
		\cline{2-4}
		  & 0 & 1 & 2 \\
		\hline
		\hline
		A & 0 & 0 & 0 \\
		\hline
		B & 2 & 2 & 2 \\
		\hline
		C & 3 & 3 & 3 \\
		\hline
		D & 1 & 2 & 0 \\
		\hline
		E & 0 & 2 & 4 \\
		\hline
	\end{tabular}
\end{table}

\begin{figure}[H]
	\caption{Grafo acíclico dirigido para el Cuadro \ref{ej2:tab1}.}
	\centering
	\begin{tikzpicture}
		\SetGraphUnit{2}
		\GraphInit[vstyle=Normal]
		\tikzset{EdgeStyle/.style={->}}
		\Vertex{A}
		\EA(A){B}
		\EA(B){C}
		\SO(A){E}
		\SO(B){D}
		\Edge(A)(B)
		\Edge(B)(C)
		\Edge[style={bend right=45}](D)(C)
	\end{tikzpicture}
\end{figure}
