\section{Ejercicio 2}

\subsection{Introducción}

El objetivo de este problema era dado un número $A$ de acciones con sus valores
en el transcurso de $D$ días devolver el mínimo número de gráficos necesarios
para graficarlas sin que hubiera intersecciones entre las visualizaciones de las
mismas.

\subsubsection{Solución propuesta}

La resolución de este ejercicio cuenta con varias etapas. La primera consistió
en modelar la entrada de forma tal que fuera posible visualizar en otros
términos el problema a resolver.

Para esto se decidió representar las acciones mediante un grafo acíclico
dirigido. El modo de construirlo fue el siguiente:

\begin{itemize}
	\item{Cada acción es representada mediante un vértice.}
	\item{Un arco saliendo de una acción $A_j$ a $A_k$ implica que el valor de
	la acción $A_j$ se encuentra acotado superiormente por el de $A_k$ en todo
	momento.}
\end{itemize}

Se puede ver fácilmente que no habrá ciclos puesto que ello implicaría que el
valor de una acción acota superior e inferiormente el de otra, lo cual sería
un absurdo.

A modo de ejemple se presenta a continuación una posible entrada con su
respectiva representación en el modelo propuesto.

\newcolumntype{M}[1]{>{\centering\arraybackslash}m{#1}}

\begin{table}[H]
	\caption{Ejemplo de entrada con $A = 5$ y $D = 3$.} \label{ej2:tab1}
	\centering
	\begin{tabular}{|c|M{2em}|M{2em}|M{2em}|}
		\hline
		\multirow{2}{4em}{\centering Acción} & \multicolumn{3}{c|}{Valor en día $i$} \\
		\cline{2-4}
		  & 0 & 1 & 2 \\
		\hline
		\hline
		A & 0 & 0 & 0 \\
		\hline
		B & 2 & 2 & 2 \\
		\hline
		C & 3 & 3 & 3 \\
		\hline
		D & 1 & 2 & 0 \\
		\hline
		E & 0 & 2 & 4 \\
		\hline
	\end{tabular}
\end{table}

\begin{figure}[H]
	\caption{Grafo acíclico dirigido para el Cuadro \ref{ej2:tab1}.} \label{ej2:fig1}
	\centering
	\begin{tikzpicture}
		\SetGraphUnit{2}
		\GraphInit[vstyle=Normal]
		\tikzset{EdgeStyle/.style={->}}
		\Vertex{A}
		\EA(A){B}
		\EA(B){C}
		\SO(A){E}
		\SO(B){D}
		\Edge(A)(B)
		\Edge[style={bend right=45}](A)(C)
		\Edge(B)(C)
		\Edge[style={bend right=45}](D)(C)
	\end{tikzpicture}
\end{figure}

Al modelar los datos del problema de esta manera, encontrar la mínima cantidad
de gráficos necesarios es equivalente al tamaño del menor cubrimiento por caminos del
grafo acíclico dirigido. Esto vale porque un camino dirigido en el modelo
planteado es un gráfico válido, donde el valor de cada acción es acotado
superiormente por sus sucesores en el camino. En particular existe un tamaño
mínimo para la cantidad de caminos dirigidos que particionan el conjunto de vértices, este
corresponderá a la cota inferior para el número de gráficos necesarios tal que
no haya intersecciones.

Es entonces cuando podemos hacer uso del siguiente teorema:

\begin{theorem}[Teorema de Dilworth] \label{ej2:teo1}
	El tamaño de la mayor anticadena es igual al tamaño del menor cubrimiento
	por caminos de un grafo acíclico dirigido transitivo.
\end{theorem}

Nuestro digrafo define un orden parcial entre las acciones, por lo tanto
encontrando el tamaño de su mayor anticadena obtenemos el de el menor
cubrimiento por caminos del digrafo.

Pero esto aún no nos dice cómo encontrar tal anticadena, lo cual nos lleva a
otro teorema:

\begin{theorem}[Teorema de K\H{o}nig] \label{ej2:teo2}
	En cualquier grafo bipartito, el número de aristas en un emparejamiento
	máximo es igual al número de vértices en una cubrimiento de vértices mínimo.
\end{theorem}

Esto en un principio pareciera no tener relación, pero es posible demostrar que
ambos teoremas son equivalentes. En particular, la prueba de que el Teorema de K\H{o}nig
implica el de Dilworth para un conjunto con orden parcial $S$ de $n$ elementos
nos da la pieza que falta para resolver el problema.

La misma, construye un grafo bipartito $G = (V_0 \cup V_1, E)$ donde $V_0 = V_1
= S$ y $(u, v)$ es una arista en $G$ cuando $u < v$ en $S$. Dado un
emparejamiento máximo de $m$ aristas, demuestra que el tamaño de la máxima
anticadena en $S$ es igual a $n - m$.

\begin{figure}[H]
	\caption{Emparejamiento máximo para el grafo bipartito
	correspondiente al digrafo en Figura \ref{ej2:fig1}} \label{ej2:fig2}
	\centering
	\begin{tikzpicture}[x=1.5cm ,y=0.5cm]
		\SetGraphUnit{2}
		\GraphInit[vstyle=Normal]
		\Vertex[L=A]{A0}
		\EA[L=A](A0){A1}
		\SO[L=B](A0){B0}
		\EA[L=B](B0){B1}
		\SO[L=C](B0){C0}
		\EA[L=C](C0){C1}
		\SO[L=D](C0){D0}
		\EA[L=D](D0){D1}
		\SO[L=E](D0){E0}
		\EA[L=E](E0){E1}
		\Edge(A0)(C1)
		\Edge(D0)(C1)
		\SetUpEdge[style={ultra thick}, color=red]
		\Edge(A0)(B1)
		\Edge(B0)(C1)
	\end{tikzpicture}
\end{figure}

Es así como llegamos al punto donde podemos concluir que para resolver el
ejercicio basta con construir tal grafo bipartito a partir del grafo acíclico
dirigido mencionado anteriormente y encontrar su emparejamiento máximo.

Este último paso, encontrar el emparejamiento máximo para un grafo bipartito, se
puede resolver mediante una red de flujo. Para construir tal red para el grafo bipartito
$G = (V_0 \cup V_1, E)$ se siguen los siguientes pasos:

\begin{itemize}
	\item{Se agrega un vértice fuente $s$.}
	\item{Se agrega un vértice sumidero $t$.}
	\item{Se agregan arcos de capacidad uno saliendo de $s$ a todo vértice en $V_0$.}
	\item{Se agregan arcos de capacidad uno saliendo de todo vértice en $V_1$ a $t$.}
	\item{Cada arista en $G$ es reemplazada por un arco de $V_0$ a $V_1$ con capacidad 1.}
\end{itemize}

El flujo máximo de esta red corresponde al tamaño de un emparejamiento máximo de
$G$.

\begin{figure}[H]
	\caption{Red de flujo con máximo flujo posible para el grafo bipartito en
	Figura \ref{ej2:fig2}}
	\centering
	\begin{tikzpicture}[x=1.5cm ,y=0.5cm]
		\SetGraphUnit{2}
		\GraphInit[vstyle=Dijkstra]
		\tikzset{EdgeStyle/.style={->}}
		\Vertex[L=A]{A0}
		\EA[L=A](A0){A1}
		\SO[L=B](A0){B0}
		\EA[L=B](B0){B1}
		\SO[L=C](B0){C0}
		\EA[L=C](C0){C1}
		\WE(C0){s}
		\EA(C1){t}
		\SO[L=D](C0){D0}
		\EA[L=D](D0){D1}
		\SO[L=E](D0){E0}
		\EA[L=E](E0){E1}
		\Edge[label=$0/1$](s)(C0)
		\Edge[label=$0/1$, style={bend right=20}](s)(D0)
		\Edge[label=$0/1$, style={bend right=35}](s)(E0)
		\Edge[label=$0/1$, style={bend right=15}](A0)(C1)
		\Edge[label=$0/1$, style={bend right=15}](D0)(C1)
		\Edge[label=$0/1$, style={bend left=35}](A1)(t)
		\Edge[label=$0/1$, style={bend right=20}](D1)(t)
		\Edge[label=$0/1$, style={bend right=35}](E1)(t)
		\SetUpEdge[style={->, ultra thick}, color=red]
		\Edge[label=$1/1$, style={bend left=35, ->, ultra thick}](s)(A0)
		\Edge[label=$1/1$, style={bend left=20, ->, ultra thick}](s)(B0)
		\Edge[label=$1/1$, style={bend right=15, ->, ultra thick}](A0)(B1)
		\Edge[label=$1/1$, style={bend right=25, ->, ultra thick}](B0)(C1)
		\Edge[label=$1/1$, style={bend left=20, ->, ultra thick}](B1)(t)
		\Edge[label=$1/1$](C1)(t)
	\end{tikzpicture}
\end{figure}

A continuación se presenta el pseudocódigo del algoritmo que resuelve el
problema:

\begin{algorithm}[H]
	\caption{Número mínimo de gráficos sin intersecciones}
	\Input{Matriz $M \in \mathbb{Z}^{A \times D}$ donde $M_{ij}$ representa el valor de la
	acción $i$ en el día $j$.}
	\Output{El mínimo número de gráficos necesarios para graficar las acciones
	sin que haya intersecciones.}

	\emph{redFlujo} $\gets$ Crear digrafo con vértices $V_0 \cup V_1 \cup \left\{s,
	t\right\}$ donde $\#(V_0) = \#(V_1) = A$ y $V_0 \cap V_1 = \emptyset$ \;
	\ForEach{vértice $u \in V_0$} {
		Agregar arco $(s, u)$ con capacidad 1 a \emph{redFlujo} \;
		\ForEach{vértice $v \in V_1$} {
			\If{$(\forall j \in [0 \dots D)) M_{uj} < M_{vj}$} {
				Agregar arco $(u, v)$ con capacidad 1 a \emph{redFlujo} \;
			}
		}
	}
	\ForEach{vértice $v \in V_1$} {
		Agregar arco $(v, t)$ con capacidad 1 a \emph{redFlujo} \;
	}
	\emph{flujoMáximo} $\gets$ Ejecutar algoritmo de Edmonds-Karp sobre \emph{redFlujo} \;
	\Return{$A$ - \emph{flujoMáximo}} \;
\end{algorithm}
