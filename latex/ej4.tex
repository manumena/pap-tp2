\section{Ejercicio 4}

Peso asignado: 9.

\subsection{Problema y modelo}

El problema plantea un escenario en el que se tienen aulas y pasillos que las conectan, permitiendo estos pasillos el recorrido en un solo sentido cada uno, no habiendo ningún pasillo que conecte un aula consigo misma ni un par de pasillos que conecten dos aulas en el mismo sentido. Esto puede modelarse mediante un grafo dirigido cuyos vértices representan las aulas y en el que hay un eje dirigido por cada pasillo, sin \textit{multiejes} (varios ejes que conecten dos vértices en un mismo sentido) ni \textit{bucles} (ejes que conecten a un vértice consigo mismo).

En el problema interesa evaluar las posibilidades de poder circular las aulas ante la necesidad de evacuar la estructura en la que se encuentran, y para ello se requiere la capacidad de evaluar cierta cantidad de consultas (de ahora en más \textit{consultas de circulación}), cada una consistiendo en un par ordenado de aulas $a$ y $b$ (notación $\langle a,b \rangle$). La solución debe poder responder para cada una si es posible recorrer los pasillos dados de modo de partir de $a$, pasar por $b$ y volver a $a$, lo cual le permitiría a personas en $a$ ir al aula $b$ para avisar de la necesidad de evacuar y volver a $a$ para recuperar sus pertenencias. En nuestro modelo, esto equivale a detectar si existe un camino dirigido entre $a$ y $b$ y otro entre $b$ y $a$ en el grafo inducido por la instancia en cuestión (también \textit{grafo de aulas y pasillos}).

Naturalmente, dada la motivación del problema que presenta el enunciado, es imperativo que las personas que circulen entre las aulas puedan luego llegar a la salida de la estructura. Como el enunciado aclara que desde cualquier aula se puede alcanzar de alguna manera esa salida, no modelamos esa parte del problema y nos concentramos únicamente en lo relativo a la circulación mencionada entre pares de aulas.

Consideramos que una consulta $\langle a,a \rangle$ siempre tiene respuesta positiva, ya que una persona en un aula determinada es capaz de alertar a los del mismo aula de la necesidad de evacuar y también de retirar sus pertenencias de esa aula.

\subsection{Solución}

\subsubsection{Idea general}

Abordaremos la solución en dos etapas. En la primera, buscaremos procesar el grafo en tiempo lineal en su cantidad de vértices y ejes para generar una estructura que, en segunda instancia, permita responder en tiempo constante cada consulta de circulación que se le haga. Notemos que, si $A$ es la cantidad de aulas, $P$ la de pasillos y $Q$ la de consultas, este enfoque naturalmente cumpliría con la complejidad requerida de $O(A+P+Q)$, incurriendo la primera etapa de procesamiento del grafo un costo de $O(A+P)$ y la segunda de respuesta a consultas uno de $O(Q)$.

Entonces, para la primera etapa, nos interesa encontrar una forma de reducir la información del grafo a una estructura que resuma las posibilidades de circulación entre las distintas aulas. Concretamente, queremos poder determinar si, para un par ordenado de aulas arbitrario, es posible ir de la primera a la segunda y volver.

Dado un grafo de aulas y pasillos, observemos que:
\begin{itemize}
    \item Si un par de aulas está en una misma componente fuertemente conexa del grafo, es posible ir de cualquiera de ellas a la otra y luego volver.
    \item Si es posible ir de un aula a otra y volver, este par de aulas conforma un subgrafo fuertemente conexo que por ende es subgrafo de alguna componente fuertemente conexa del grafo.
\end{itemize}

Esta última observación establece una equivalencia entre la positividad de la respuesta correcta a una consulta de circulación y la pertenencia de las aulas involucradas en ella a una misma componente fuertemente conexa del grafo de aulas y pasillos. De esta forma, la respuesta a la consulta $\langle a,b \rangle$ debe ser positiva si $a$ y $b$ pertenecen a la misma componente conexa y negativa si no.

En conclusión, en la primera etapa nos bastará con construir en tiempo lineal, a partir de un grafo de aulas y pasillos, una estructura que permita consultar en tiempo constante si un par de aulas corresponden a una misma componente fuertemente conexa de ese grafo. Luego, en la segunda etapa, sencillamente evaluaremos cada una de las $Q$ consultas en tiempo constante utilizando esa estructura.

\subsubsection{Algoritmo}

La primera etapa consiste esencialmente en la aplicación al grafo de un
algoritmo lineal de detección de componentes fuertemente conexas. Elegimos
utilizar el \textit{Algoritmo de Kosaraju} para ese propósito, implementándolo
de forma que asigne a cada componente fuertemente conexa que detecta una
identificación unívoca del tipo $scc\_id$ y devuelva un mapa $\langle v\_id,
scc\_id \rangle$ que asocie a cada vértice, identificado unívocamente por un
valor del tipo $v\_id$, la componente fuertemente conexa a la que pertenece. Los
mapas utilizados en nuestro algoritmo se consideran siempre de construcción en
tiempo lineal en su cantidad de claves y de consulta y actualización en tiempo
constante. Por su parte, los grafos se representan mediante listas de adyacencia
que identifican los vértices por valores únicos de $v\_id$.

\bigskip

\begin{algorithm}[H]
	\caption{Kosaraju}
	\Input{$G$: grafo dirigido}
    \Output{$M$: mapa $\langle v\_id, scc\_id \rangle$ }

    $\mathit{secuenciaDeCierre} \gets$ secuenciar los vértices en orden decreciente de cierre corriendo \textit{DFS} sobre G \;

    $\mathit{G^T} \gets$ trasponer $G$ mediante un \textit{DFS} (ver subsección sobre complejidad) \;

    $\mathit{M} \gets \mathit{mapearVerticesConSCCs(G^T)}$

    \Return{$M$}
\end{algorithm}

\bigskip

\begin{algorithm}[H]
    \caption{mapearVerticesConSCCs}
    \Input{$G$: grafo dirigido}
    \Output{$M$: mapa $\langle v\_id, scc\_id \rangle$ }

    $\mathit{mapeado} \gets$ mapa $\langle v\_id, bool \rangle$ que se inicializa asociando cada vértice de G a \textit{false}\;

    $id \gets$ obtenerIdUnicoDeSCC() \;
    \For {$v$ en $V(G)$} {
        \If {el valor de  $v$ en $mapeados$ es \textit{false}} {
            Explorar mediante \textit{DFS} sólo un árbol de $G$ empezando en $v$ como su raíz, marcando cada vértice descubierto con $true$ en $mapeados$ y asociándole en $M$ el valor $id$ \;
            $id \gets$ obtenerIdUnicoDeSCC() \;
        }
    }

    \Return{$M$}
\end{algorithm}

\medskip

siendo \textit{obtenerIdUnicoDeSCC()} una función que provee valores únicos de tipo $scc\_id$ con cada invocación en tiempo constante.

\medskip

En la segunda etapa, queda sencillamente usar el mapa construido por Kosaraju para responder las consultas de circulación. El algoritmo completo, incluyendo la segunda etapa, es el siguiente.

\bigskip

\begin{algorithm}[H]
	\caption{ResponderConsultas}
	\Input{
        $G$: grafo dirigido representado con listas de adyacencia, $Q$: secuencia de consultas de circulación $\langle a, b \rangle$
    }

    $M \gets \mathit{Kosaraju(G)}$ \;

    \For {$\langle a, b \rangle$ en $Q$} {
        \If {$M$ en $a$ es igual a $M$ en $b$} {
            responder afirmativamente a $\langle a, b \rangle$
        } \Else {
            responder negativamente a $\langle a, b \rangle$
        }
    }
\end{algorithm}

\subsubsection{Complejidad}

Sea $G$ el grafo de aulas y pasillos de alguna instancia del problema con $Q$ consultas de circulación. Sea $A$ cantidad de vértices de $G$ y $P$ su cantidad de ejes.

En nuestra implementación de \textit{Kosaraju} (\textbf{Algoritmo 1}), en la línea \texttt{1} se construye un vector de los vértices secuencialmente según el orden de finalización de los mismos en una exploración \textit{DFS}, de manera análoga a lo visto en clase. Por lo tanto, esta línea agrega un tiempo de $O(A+P)$. En la línea \texttt{2}, creamos el grafo traspuesto $G^T$ de $G$, corriendo sobre $G$ un \textit{DFS} que, en la etapa de cierre de cada vértice $v$, agrega en $G^T$ el eje $(w,v)$ por cada eje $(v,w)$ saliente de $v$. Es decir que, al cerrar cada vértice $v$, suma un costo de $O(out(v))$ siendo $out(v)$ el \textit{out-degree} de $v$. Por ende, los pasos de construcción de $G^T$ agregan al \textit{DFS} un costo total de $O(P)$, dado que cada vértice es cerrado exactamente una vez por el \textit{DFS} y $\sum\limits_{v \in V(G)} out(v) = P$, obteniendo un nuevo aporte de $O(A+P)$ operaciones. Por último, en la tercera línea es donde se explora $G^T$ mediante \textit{DFS} según el orden calculado en la primera línea. La única modificación al paso análogo de la versión del algoritmo dada en clase es el agregado en tiempo constante de cada vértice al mapa correspondiente en lugar de su impresión, con lo que alcanza para afirmar que el costo temporal es el mismo de $O(A+P)$. En conclusión, tenemos que este algoritmo tiene complejidad temporal $O(A+P)$.

Finalmente, en el algoritmo completo que resuelve el problema (\textbf{Algoritmo 3}), tenemos una invocación a la implementación de \textit{Kosaraju} del \textbf{Algoritmo 1} de costo $O(A+P)$, y luego $Q$ iteraciones de respuesta a cada consulta en tiempo constante en base al mapa resultante de la corrida de \textit{Kosaraju}, dando en total una complejidad temporal de $O(A+P+Q)$.

\subsubsection{Detalles de implementación}

Como antes de resolver cada instancia se conoce la cantidad $A$ de vértices que tendrá el grafo de aulas y pasillos correspondientes, para los valores de tipo \textit{v\_id} usamos enteros de $0$ a $A-1$ e implementamos los mapas utilizados por nuestro algoritmo como arreglos de $A$ elementos de tipo $T$ que contienen en la posición $i$ el valor asociado al vértice identificado por $i$. De este modo, logramos la complejidad pretendida de inicialización en tiempo lineal en cantidad de claves y de consulta y modificación en $O(1)$.

\subsection{Casos de prueba}

Se utilizaron instancias de prueba generadas manualmente que corresponden a las siguientes familias de instancias:

\begin{itemize}
    \item Grafos sin ninguna componente fuertemente conexa, consultando por distintos pares $\langle a, b \rangle$, tanto con $a \neq b$ como con $a = b$. Las únicas que deben responderse afirmativamente son las del segundo tipo.
    \item Grafos con una sola componente fuertemente conexa y cantidades de vértices mayores a 1, consultando por distintos pares $\langle a, b \rangle$, tanto con $a \neq b$ como con $a = b$. Deben responderse afirmativamente todas las consultas.
    \item Grafos con distintas componentes fuertemente conexas desconectadas entre sí, con cantidades de vértices mayores o iguales a 1, consultando por distintos pares $\langle a, b \rangle$, tanto con $a \neq b$ como con $a = b$. Deben responderse positivamente si y sólo si pertenecen a una misma componente.
    \item Grafos con distintas componentes fuertemente conexas conectadas entre sí, consultando por distintos pares $\langle a, b \rangle$ tales que $a$ y $b$ corresponden a la misma componente, tales que $a$ y $b$ pertenecen a distintas componentes y se puede ir de $a$ a $b$ pero no volver, y tales que $a$ y $b$ pertenecen a distintas componentes y se puede volver de $b$ a $a$ pero no ir. Deben responderse positivamente si y sólo si pertenecen a una misma componente.
\end{itemize}

Además, se usaron grafos de dimensiones en el orden de las cotas para testeo propuestas en el enunciado con el objetivo de corroborar el funcionamiento correcto de la implementación ante un mayor volumen de datos. Se utilizaron grafos con una sola componente fuertemente conexa en los cuales, según la teoría, cualquier consulta debe ser respondida afirmativamente.
